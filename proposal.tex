% Options for packages loaded elsewhere
\PassOptionsToPackage{unicode}{hyperref}
\PassOptionsToPackage{hyphens}{url}
\PassOptionsToPackage{dvipsnames,svgnames,x11names}{xcolor}
%
\documentclass[
  letterpaper,
]{article}

\usepackage{amsmath,amssymb}
\usepackage{iftex}
\ifPDFTeX
  \usepackage[T1]{fontenc}
  \usepackage[utf8]{inputenc}
  \usepackage{textcomp} % provide euro and other symbols
\else % if luatex or xetex
  \usepackage{unicode-math}
  \defaultfontfeatures{Scale=MatchLowercase}
  \defaultfontfeatures[\rmfamily]{Ligatures=TeX,Scale=1}
\fi
\usepackage{lmodern}
\ifPDFTeX\else  
    % xetex/luatex font selection
\fi
% Use upquote if available, for straight quotes in verbatim environments
\IfFileExists{upquote.sty}{\usepackage{upquote}}{}
\IfFileExists{microtype.sty}{% use microtype if available
  \usepackage[]{microtype}
  \UseMicrotypeSet[protrusion]{basicmath} % disable protrusion for tt fonts
}{}
\makeatletter
\@ifundefined{KOMAClassName}{% if non-KOMA class
  \IfFileExists{parskip.sty}{%
    \usepackage{parskip}
  }{% else
    \setlength{\parindent}{0pt}
    \setlength{\parskip}{6pt plus 2pt minus 1pt}}
}{% if KOMA class
  \KOMAoptions{parskip=half}}
\makeatother
\usepackage{xcolor}
\usepackage[margin=1in]{geometry}
\setlength{\emergencystretch}{3em} % prevent overfull lines
\setcounter{secnumdepth}{-\maxdimen} % remove section numbering
% Make \paragraph and \subparagraph free-standing
\ifx\paragraph\undefined\else
  \let\oldparagraph\paragraph
  \renewcommand{\paragraph}[1]{\oldparagraph{#1}\mbox{}}
\fi
\ifx\subparagraph\undefined\else
  \let\oldsubparagraph\subparagraph
  \renewcommand{\subparagraph}[1]{\oldsubparagraph{#1}\mbox{}}
\fi

\usepackage{color}
\usepackage{fancyvrb}
\newcommand{\VerbBar}{|}
\newcommand{\VERB}{\Verb[commandchars=\\\{\}]}
\DefineVerbatimEnvironment{Highlighting}{Verbatim}{commandchars=\\\{\}}
% Add ',fontsize=\small' for more characters per line
\usepackage{framed}
\definecolor{shadecolor}{RGB}{241,243,245}
\newenvironment{Shaded}{\begin{snugshade}}{\end{snugshade}}
\newcommand{\AlertTok}[1]{\textcolor[rgb]{0.68,0.00,0.00}{#1}}
\newcommand{\AnnotationTok}[1]{\textcolor[rgb]{0.37,0.37,0.37}{#1}}
\newcommand{\AttributeTok}[1]{\textcolor[rgb]{0.40,0.45,0.13}{#1}}
\newcommand{\BaseNTok}[1]{\textcolor[rgb]{0.68,0.00,0.00}{#1}}
\newcommand{\BuiltInTok}[1]{\textcolor[rgb]{0.00,0.23,0.31}{#1}}
\newcommand{\CharTok}[1]{\textcolor[rgb]{0.13,0.47,0.30}{#1}}
\newcommand{\CommentTok}[1]{\textcolor[rgb]{0.37,0.37,0.37}{#1}}
\newcommand{\CommentVarTok}[1]{\textcolor[rgb]{0.37,0.37,0.37}{\textit{#1}}}
\newcommand{\ConstantTok}[1]{\textcolor[rgb]{0.56,0.35,0.01}{#1}}
\newcommand{\ControlFlowTok}[1]{\textcolor[rgb]{0.00,0.23,0.31}{#1}}
\newcommand{\DataTypeTok}[1]{\textcolor[rgb]{0.68,0.00,0.00}{#1}}
\newcommand{\DecValTok}[1]{\textcolor[rgb]{0.68,0.00,0.00}{#1}}
\newcommand{\DocumentationTok}[1]{\textcolor[rgb]{0.37,0.37,0.37}{\textit{#1}}}
\newcommand{\ErrorTok}[1]{\textcolor[rgb]{0.68,0.00,0.00}{#1}}
\newcommand{\ExtensionTok}[1]{\textcolor[rgb]{0.00,0.23,0.31}{#1}}
\newcommand{\FloatTok}[1]{\textcolor[rgb]{0.68,0.00,0.00}{#1}}
\newcommand{\FunctionTok}[1]{\textcolor[rgb]{0.28,0.35,0.67}{#1}}
\newcommand{\ImportTok}[1]{\textcolor[rgb]{0.00,0.46,0.62}{#1}}
\newcommand{\InformationTok}[1]{\textcolor[rgb]{0.37,0.37,0.37}{#1}}
\newcommand{\KeywordTok}[1]{\textcolor[rgb]{0.00,0.23,0.31}{#1}}
\newcommand{\NormalTok}[1]{\textcolor[rgb]{0.00,0.23,0.31}{#1}}
\newcommand{\OperatorTok}[1]{\textcolor[rgb]{0.37,0.37,0.37}{#1}}
\newcommand{\OtherTok}[1]{\textcolor[rgb]{0.00,0.23,0.31}{#1}}
\newcommand{\PreprocessorTok}[1]{\textcolor[rgb]{0.68,0.00,0.00}{#1}}
\newcommand{\RegionMarkerTok}[1]{\textcolor[rgb]{0.00,0.23,0.31}{#1}}
\newcommand{\SpecialCharTok}[1]{\textcolor[rgb]{0.37,0.37,0.37}{#1}}
\newcommand{\SpecialStringTok}[1]{\textcolor[rgb]{0.13,0.47,0.30}{#1}}
\newcommand{\StringTok}[1]{\textcolor[rgb]{0.13,0.47,0.30}{#1}}
\newcommand{\VariableTok}[1]{\textcolor[rgb]{0.07,0.07,0.07}{#1}}
\newcommand{\VerbatimStringTok}[1]{\textcolor[rgb]{0.13,0.47,0.30}{#1}}
\newcommand{\WarningTok}[1]{\textcolor[rgb]{0.37,0.37,0.37}{\textit{#1}}}

\providecommand{\tightlist}{%
  \setlength{\itemsep}{0pt}\setlength{\parskip}{0pt}}\usepackage{longtable,booktabs,array}
\usepackage{calc} % for calculating minipage widths
% Correct order of tables after \paragraph or \subparagraph
\usepackage{etoolbox}
\makeatletter
\patchcmd\longtable{\par}{\if@noskipsec\mbox{}\fi\par}{}{}
\makeatother
% Allow footnotes in longtable head/foot
\IfFileExists{footnotehyper.sty}{\usepackage{footnotehyper}}{\usepackage{footnote}}
\makesavenoteenv{longtable}
\usepackage{graphicx}
\makeatletter
\def\maxwidth{\ifdim\Gin@nat@width>\linewidth\linewidth\else\Gin@nat@width\fi}
\def\maxheight{\ifdim\Gin@nat@height>\textheight\textheight\else\Gin@nat@height\fi}
\makeatother
% Scale images if necessary, so that they will not overflow the page
% margins by default, and it is still possible to overwrite the defaults
% using explicit options in \includegraphics[width, height, ...]{}
\setkeys{Gin}{width=\maxwidth,height=\maxheight,keepaspectratio}
% Set default figure placement to htbp
\makeatletter
\def\fps@figure{htbp}
\makeatother
\newlength{\cslhangindent}
\setlength{\cslhangindent}{1.5em}
\newlength{\csllabelwidth}
\setlength{\csllabelwidth}{3em}
\newlength{\cslentryspacingunit} % times entry-spacing
\setlength{\cslentryspacingunit}{\parskip}
\newenvironment{CSLReferences}[2] % #1 hanging-ident, #2 entry spacing
 {% don't indent paragraphs
  \setlength{\parindent}{0pt}
  % turn on hanging indent if param 1 is 1
  \ifodd #1
  \let\oldpar\par
  \def\par{\hangindent=\cslhangindent\oldpar}
  \fi
  % set entry spacing
  \setlength{\parskip}{#2\cslentryspacingunit}
 }%
 {}
\usepackage{calc}
\newcommand{\CSLBlock}[1]{#1\hfill\break}
\newcommand{\CSLLeftMargin}[1]{\parbox[t]{\csllabelwidth}{#1}}
\newcommand{\CSLRightInline}[1]{\parbox[t]{\linewidth - \csllabelwidth}{#1}\break}
\newcommand{\CSLIndent}[1]{\hspace{\cslhangindent}#1}


\usepackage{booktabs}
\usepackage{longtable}
\usepackage{array}
\usepackage{multirow}
\usepackage{wrapfig}
\usepackage{float}
\usepackage{colortbl}
\usepackage{pdflscape}
\usepackage{tabu}
\usepackage{threeparttable}
\usepackage{threeparttablex}
\usepackage[normalem]{ulem}
\usepackage{makecell}
\usepackage{xcolor}
\makeatletter
\makeatother
\makeatletter
\makeatother
\makeatletter
\@ifpackageloaded{caption}{}{\usepackage{caption}}
\AtBeginDocument{%
\ifdefined\contentsname
  \renewcommand*\contentsname{Table of contents}
\else
  \newcommand\contentsname{Table of contents}
\fi
\ifdefined\listfigurename
  \renewcommand*\listfigurename{List of Figures}
\else
  \newcommand\listfigurename{List of Figures}
\fi
\ifdefined\listtablename
  \renewcommand*\listtablename{List of Tables}
\else
  \newcommand\listtablename{List of Tables}
\fi
\ifdefined\figurename
  \renewcommand*\figurename{Figure}
\else
  \newcommand\figurename{Figure}
\fi
\ifdefined\tablename
  \renewcommand*\tablename{Table}
\else
  \newcommand\tablename{Table}
\fi
}
\@ifpackageloaded{float}{}{\usepackage{float}}
\floatstyle{ruled}
\@ifundefined{c@chapter}{\newfloat{codelisting}{h}{lop}}{\newfloat{codelisting}{h}{lop}[chapter]}
\floatname{codelisting}{Listing}
\newcommand*\listoflistings{\listof{codelisting}{List of Listings}}
\makeatother
\makeatletter
\@ifpackageloaded{caption}{}{\usepackage{caption}}
\@ifpackageloaded{subcaption}{}{\usepackage{subcaption}}
\makeatother
\makeatletter
\@ifpackageloaded{tcolorbox}{}{\usepackage[skins,breakable]{tcolorbox}}
\makeatother
\makeatletter
\@ifundefined{shadecolor}{\definecolor{shadecolor}{rgb}{.97, .97, .97}}
\makeatother
\makeatletter
\makeatother
\makeatletter
\makeatother
\ifLuaTeX
  \usepackage{selnolig}  % disable illegal ligatures
\fi
\IfFileExists{bookmark.sty}{\usepackage{bookmark}}{\usepackage{hyperref}}
\IfFileExists{xurl.sty}{\usepackage{xurl}}{} % add URL line breaks if available
\urlstyle{same} % disable monospaced font for URLs
\hypersetup{
  pdftitle={Variance All the Way Down: Exploring the Impact of RNA-Seq Pipeline Choices on Differential Expression Variance},
  pdfauthor={Hunter Schuler and Art Tay},
  colorlinks=true,
  linkcolor={blue},
  filecolor={Maroon},
  citecolor={Blue},
  urlcolor={Blue},
  pdfcreator={LaTeX via pandoc}}

\title{\textbf{Variance All the Way Down:} Exploring the Impact of
RNA-Seq Pipeline Choices on Differential Expression Variance}
\author{Hunter Schuler and Art Tay}
\date{}

\begin{document}
\maketitle
\ifdefined\Shaded\renewenvironment{Shaded}{\begin{tcolorbox}[boxrule=0pt, frame hidden, borderline west={3pt}{0pt}{shadecolor}, interior hidden, enhanced, breakable, sharp corners]}{\end{tcolorbox}}\fi

\hypertarget{abstract}{%
\section{Abstract}\label{abstract}}

In the realm of RNA-Seq research, rigorous data preprocessing is a
critical foundation for meaningful analysis. Despite its importance,
this preprocessing involves numerous stages, each introducing potential
sources of variance. While previous studies have examined the overall
variance between entire RNA-Seq pipelines, (Arora et al. 2020) (Tong et
al. 2020), (Vieth et al. 2019), the impact of individual stages remains
less understood. We propose a comprehensive investigation into the
variance introduced at each stage of RNA-Seq preprocessing. Our goal is
to quantify these variances, study their distributions, and understand
their statistical implications on downstream modeling. This will include
exploring the multitude of decisions researchers face --- from quality
control to normalization --- and evaluating how these choices propagate
uncertainty through the analysis. Of particular interest is whether
variance amplifies due to interactions between decisions made at
different stages. By modeling these interactions, we aim to identify
cases where suboptimal combinations of preprocessing choices exacerbate
variability, potentially distorting biological interpretations. This
work aims to provide researchers with actionable insights to mitigate
preprocessing-induced variance, ultimately enhancing the reliability and
reproducibility of RNA-Seq studies.

\hypertarget{question-of-interest}{%
\section{Question of Interest}\label{question-of-interest}}

\begin{quote}
How do discretionary choices made during RNA-Seq pipeline processing,
such as 'fasterq-dump` options, quality filtering threshold, the choice
of aligner, and normalization method impact the variance of differential
expression results?
\end{quote}

We hypothesize that differences in these choices will lead to
significant variance in DE results, particularly in terms of how
consistently differentially expressed genes are identified across
pipeline variations. This variance could introduce substantial
uncertainty into the interpretation of gene expression data, influencing
biological conclusions.

\hypertarget{ideas-for-exploration}{%
\section{Ideas for Exploration}\label{ideas-for-exploration}}

\hypertarget{regression-analysis}{%
\subsection{Regression Analysis}\label{regression-analysis}}

\hypertarget{tbl-1}{}
\begin{longtable}[t]{llll}
\caption{\label{tbl-1}Basic RNA-Seq Differential Analysis End-to-End Pipeline }\tabularnewline

\toprule
Pipeline Steps & Software & Options & Choices\\
\midrule
1. Pull SRA data from the NIH. & prefetch & NA & NA\\
 &  &  & \\
2. Compute quality scores. & fasterq-dump & --skip-technical & Boolean\\
 &  & --threads X & Integer\\
 &  &  & \\
3. Filter low quality reads. & fastp & --qualified\_quality\_phred X & Integer\\
 &  & --length\_required X & Integer\\
 &  &  & \\
4. Trim excess bases. & fastp & --trim\_poly\_g & Boolean\\
 &  & --trim\_ploy\_x & Boolean\\
 &  &  & \\
5. Align and count genes. & Various & Default & Salmon, Kallisto\\
 &  &  & \\
6. Count normalization. & edgeR & calcNormFactors(method='X') & TMM, RLE, upperquartile\\
 &  &  & \\
7. Differential expression analysis. & edgeR & Default & NA\\
\bottomrule
\end{longtable}

Assume there are \(n\) samples of \(G\) gene counts. Let \(B_{gi}\)
denote the count for gene \(g\) in sample \(i\) reported to the NIH
database, and let \(C_{giX}\) denote the count obtained from pipeline
with choices \(X\). Similar let \(D_g\) and \(E_{gX}\) denote the
p-values obtained from \texttt{edgeR}. Now,\\
\begin{equation}
    Y_{1X} = \frac{1}{nG} \sum_{i=1}^n \sum_{g=1}^G (C_{giX} - B_{gi})^2  
\end{equation} and \begin{equation} 
    Y_{2X} = \frac 1 G \sum_{g=1}^G (E_{gX} - D_g)^2  
\end{equation} Our primary analysis will focus on the two following
regression models: \begin{equation}
    Y_{1X} = \beta_0 + \sum_{i=1}^p \beta_i X_i + 
        \sum_{1\leq i < j \leq p} \beta_{ij}(X_i \times X_j) + \epsilon
\end{equation} and \begin{equation}
    Y_{2X} = \beta_0 + \sum_{i=1}^p \beta_i X_i + 
        \sum_{1\leq i < j \leq p} \beta_{ij}(X_i \times X_j) + \epsilon
\end{equation} where \(p\) is the number of pipeline choices from
Table~\ref{tbl-1}. The first model studies the effect of each pipeline
choice, include all pairwise interactions, on the average square
deviation from the official NIH count matrix. The second model does the
same, but for the p-values from a differential expression analysis.

\hypertarget{code-availability-reproducibility}{%
\subsection{Code Availability \&
Reproducibility}\label{code-availability-reproducibility}}

All code will be open sourced and available on GitHub. The repository
will contain a \texttt{docker-compose.ylm} file that should allow the
exact development environment to be recreated. An \texttt{R} package is
insufficient due to a heavy use of command line tools, some of which are
platform dependent. All code targets \texttt{linux} and builds off of
the official \texttt{Bioconductor} docker image.

\hypertarget{preliminary-results}{%
\section{Preliminary Results}\label{preliminary-results}}

\hypertarget{quality-score-variance-due-to-fasterq-dump-options}{%
\subsection{Quality Score Variance Due to Fasterq-dump
Options}\label{quality-score-variance-due-to-fasterq-dump-options}}

Script that runs fasterq-dump with different options.

\begin{Shaded}
\begin{Highlighting}[]
\CommentTok{\#!/bin/bash}

\VariableTok{ACCESSION\_LIST}\OperatorTok{=}\StringTok{"SRR\_Acc\_List.txt"}

\FunctionTok{mkdir} \AttributeTok{{-}p}\NormalTok{ SRA}
\FunctionTok{mkdir} \AttributeTok{{-}p}\NormalTok{ dump\_1 dump\_2 dump\_3}

\ControlFlowTok{while} \BuiltInTok{read} \AttributeTok{{-}r} \VariableTok{SRR\_ID}\KeywordTok{;} \ControlFlowTok{do}

    \BuiltInTok{echo} \StringTok{"Processing SRR ID: }\VariableTok{$SRR\_ID}\StringTok{"}

    \ExtensionTok{prefetch} \VariableTok{$SRR\_ID} \AttributeTok{{-}{-}output{-}directory}\NormalTok{ SRA}

    \CommentTok{\# Option 1: Default}
    \ExtensionTok{fasterq{-}dump}\NormalTok{ SRA/}\VariableTok{$SRR\_ID}\NormalTok{/}\VariableTok{$SRR\_ID}\NormalTok{.sra }\AttributeTok{{-}{-}outdir}\NormalTok{ dump\_1 }\AttributeTok{{-}{-}split{-}files} \AttributeTok{{-}{-}progress} 

    \CommentTok{\# Option 2: Skip Technical }
    \ExtensionTok{fasterq{-}dump}\NormalTok{ SRA/}\VariableTok{$SRR\_ID}\NormalTok{/}\VariableTok{$SRR\_ID}\NormalTok{.sra }\AttributeTok{{-}{-}outdir}\NormalTok{ dump\_2 }\AttributeTok{{-}{-}split{-}files} \AttributeTok{{-}{-}progress} \AttributeTok{{-}{-}skip{-}technical}

    \CommentTok{\# Option 3: 10 Threads}
    \ExtensionTok{fasterq{-}dump}\NormalTok{ SRA/}\VariableTok{$SRR\_ID}\NormalTok{/}\VariableTok{$SRR\_ID}\NormalTok{.sra }\AttributeTok{{-}{-}outdir}\NormalTok{ dump\_3 }\AttributeTok{{-}{-}split{-}files} \AttributeTok{{-}{-}progress} \AttributeTok{{-}{-}threads}\NormalTok{ 10}

\ControlFlowTok{done} \OperatorTok{\textless{}} \StringTok{"}\VariableTok{$ACCESSION\_LIST}\StringTok{"}

\BuiltInTok{echo} \StringTok{"Processing complete!"}
\end{Highlighting}
\end{Shaded}

\texttt{R} code to analyze any differences in quality scores.

\begin{Shaded}
\begin{Highlighting}[]
\ControlFlowTok{if}\NormalTok{ (}\SpecialCharTok{!}\FunctionTok{require}\NormalTok{(}\StringTok{"BiocManager"}\NormalTok{, }\AttributeTok{quietly =} \ConstantTok{TRUE}\NormalTok{))}
    \FunctionTok{install.packages}\NormalTok{(}\StringTok{"BiocManager"}\NormalTok{)}

\NormalTok{BiocManager}\SpecialCharTok{::}\FunctionTok{install}\NormalTok{(}\StringTok{"ShortRead"}\NormalTok{)}
\NormalTok{BiocManager}\SpecialCharTok{::}\FunctionTok{install}\NormalTok{(}\StringTok{"Rsubread"}\NormalTok{)}
\end{Highlighting}
\end{Shaded}

\begin{Shaded}
\begin{Highlighting}[]
\FunctionTok{library}\NormalTok{(ShortRead)}

\NormalTok{sample\_1\_fq\_1 }\OtherTok{\textless{}{-}} \FunctionTok{readFastq}\NormalTok{(}\StringTok{"./data/dump\_1/SRR31476642.fastq"}\NormalTok{) }
\NormalTok{sample\_1\_fq\_2 }\OtherTok{\textless{}{-}} \FunctionTok{readFastq}\NormalTok{(}\StringTok{"./data/dump\_2/SRR31476642.fastq"}\NormalTok{)}
\NormalTok{sample\_1\_fq\_3 }\OtherTok{\textless{}{-}} \FunctionTok{readFastq}\NormalTok{(}\StringTok{"./data/dump\_3/SRR31476642.fastq"}\NormalTok{)}
\end{Highlighting}
\end{Shaded}

\begin{Shaded}
\begin{Highlighting}[]
\NormalTok{sample\_1\_fq\_1\_qual }\OtherTok{\textless{}{-}} \FunctionTok{as}\NormalTok{(}\FunctionTok{quality}\NormalTok{(sample\_1\_fq\_1), }\StringTok{"matrix"}\NormalTok{)}
\NormalTok{sample\_1\_fq\_2\_qual }\OtherTok{\textless{}{-}} \FunctionTok{as}\NormalTok{(}\FunctionTok{quality}\NormalTok{(sample\_1\_fq\_2), }\StringTok{"matrix"}\NormalTok{)}
\NormalTok{sample\_1\_fq\_3\_qual }\OtherTok{\textless{}{-}} \FunctionTok{as}\NormalTok{(}\FunctionTok{quality}\NormalTok{(sample\_1\_fq\_3), }\StringTok{"matrix"}\NormalTok{)}

\NormalTok{sample\_1\_fq\_13\_qual\_diff }\OtherTok{\textless{}{-}}\NormalTok{ sample\_1\_fq\_1\_qual }\SpecialCharTok{{-}}\NormalTok{ sample\_1\_fq\_3\_qual}
\NormalTok{sample\_1\_fq\_12\_qual\_diff }\OtherTok{\textless{}{-}}\NormalTok{ sample\_1\_fq\_1\_qual }\SpecialCharTok{{-}}\NormalTok{ sample\_1\_fq\_2\_qual}
\end{Highlighting}
\end{Shaded}

\begin{Shaded}
\begin{Highlighting}[]
\FunctionTok{mean}\NormalTok{(sample\_1\_fq\_13\_qual\_diff)}
\FunctionTok{mean}\NormalTok{(sample\_1\_fq\_12\_qual\_diff)}
\end{Highlighting}
\end{Shaded}

None of the \texttt{fasterq-dump} options we tested resulted in
differing quality scores.

\hypertarget{sampling-count-matrices-under-different-pipeline-choices}{%
\subsection{Sampling Count Matrices under Different Pipeline
Choices}\label{sampling-count-matrices-under-different-pipeline-choices}}

Below is a script to sample 1 count matrix from an random pipeline that
uses salmon as its aligner.

\begin{Shaded}
\begin{Highlighting}[]
\CommentTok{\#!/bin/bash}

\VariableTok{THREADS}\OperatorTok{=}\NormalTok{8}
\VariableTok{SALMON\_INDEX}\OperatorTok{=}\StringTok{"salmon\_index"}
\VariableTok{TX2GENE}\OperatorTok{=}\StringTok{"tx2gene.csv"}
\VariableTok{SRR\_LIST}\OperatorTok{=}\StringTok{"../SRR\_Acc\_List.txt"}  \CommentTok{\# File containing SRR IDs, one per line}
\VariableTok{FASTQ\_DIR}\OperatorTok{=}\StringTok{"../fastq\_data"}
\VariableTok{OUTPUT\_DIR}\OperatorTok{=}\StringTok{"salmon\_count\_matrices"}
\VariableTok{R\_SCRIPT}\OperatorTok{=}\StringTok{"generate\_salmon\_count\_matrix.R"}

\FunctionTok{mkdir} \AttributeTok{{-}p} \StringTok{"}\VariableTok{$OUTPUT\_DIR}\StringTok{"} \StringTok{"}\VariableTok{$FASTQ\_DIR}\StringTok{"}

\VariableTok{TEMP\_DIR}\OperatorTok{=}\VariableTok{$(}\FunctionTok{mktemp} \AttributeTok{{-}d}\VariableTok{)}

\VariableTok{PARAMS\_STR}\OperatorTok{=}\StringTok{""}
\CommentTok{\# Randomly select parameters. }
\VariableTok{QUAL\_PHRED}\OperatorTok{=}\VariableTok{$(}\FunctionTok{awk} \AttributeTok{{-}v}\NormalTok{ min=20 }\AttributeTok{{-}v}\NormalTok{ max=30 }\StringTok{\textquotesingle{}BEGIN\{srand(); print int(min+rand()*(max{-}min+1))\}\textquotesingle{}}\VariableTok{)}
\VariableTok{LEN\_REQ}\OperatorTok{=}\VariableTok{$(}\FunctionTok{awk} \AttributeTok{{-}v}\NormalTok{ min=30 }\AttributeTok{{-}v}\NormalTok{ max=50 }\StringTok{\textquotesingle{}BEGIN\{srand(); print int(min+rand()*(max{-}min+1))\}\textquotesingle{}}\VariableTok{)}

\VariableTok{TRIM\_G}\OperatorTok{=}\VariableTok{$((RANDOM} \OperatorTok{\%} \DecValTok{2}\VariableTok{))}
\VariableTok{TRIM\_X}\OperatorTok{=}\VariableTok{$((RANDOM} \OperatorTok{\%} \DecValTok{2}\VariableTok{))}

\VariableTok{PARAMS\_STR}\OperatorTok{=}\StringTok{"Q}\VariableTok{$\{QUAL\_PHRED\}}\StringTok{\_L}\VariableTok{$\{LEN\_REQ\}}\StringTok{\_G}\VariableTok{$\{TRIM\_G\}}\StringTok{\_X}\VariableTok{$\{TRIM\_X\}}\StringTok{"}

\BuiltInTok{echo} \StringTok{"Processed files with QUAL\_PHRED=}\VariableTok{$QUAL\_PHRED}\StringTok{, LEN\_REQ=}\VariableTok{$LEN\_REQ}\StringTok{, TRIM\_G=}\VariableTok{$TRIM\_G}\StringTok{, TRIM\_X=}\VariableTok{$TRIM\_X}\StringTok{"}

\ExtensionTok{conda}\NormalTok{ install }\AttributeTok{{-}c}\NormalTok{ bioconda fastp salmon}

\ControlFlowTok{for}\NormalTok{ FILE }\KeywordTok{in} \StringTok{"}\VariableTok{$FASTQ\_DIR}\StringTok{"}\NormalTok{/}\PreprocessorTok{*}\NormalTok{.fastq}\KeywordTok{;} \ControlFlowTok{do} 

    \BuiltInTok{echo} \StringTok{"Now processing }\VariableTok{$FILE}\StringTok{..."}

    \VariableTok{BASENAME}\OperatorTok{=}\VariableTok{$(}\FunctionTok{basename} \StringTok{"}\VariableTok{$FILE}\StringTok{"}\NormalTok{ .fastq}\VariableTok{)}

    \CommentTok{\# Filter and trim based on sampled parameters. }
    \ExtensionTok{fastp} \DataTypeTok{\textbackslash{}}
        \AttributeTok{{-}{-}in1} \StringTok{"}\VariableTok{$FILE}\StringTok{"} \DataTypeTok{\textbackslash{}}
        \AttributeTok{{-}{-}qualified\_quality\_phred} \StringTok{"}\VariableTok{$QUAL\_PHRED}\StringTok{"} \DataTypeTok{\textbackslash{}}
        \AttributeTok{{-}{-}length\_required} \StringTok{"}\VariableTok{$LEN\_REQ}\StringTok{"} \DataTypeTok{\textbackslash{}}
        \VariableTok{$(} \BuiltInTok{[} \StringTok{"}\VariableTok{$TRIM\_G}\StringTok{"} \OtherTok{{-}eq}\NormalTok{ 1 }\BuiltInTok{]} \KeywordTok{\&\&} \BuiltInTok{echo} \StringTok{"{-}{-}trim\_poly\_g"} \VariableTok{)} \DataTypeTok{\textbackslash{}}
        \VariableTok{$(} \BuiltInTok{[} \StringTok{"}\VariableTok{$TRIM\_X}\StringTok{"} \OtherTok{{-}eq}\NormalTok{ 1 }\BuiltInTok{]} \KeywordTok{\&\&} \BuiltInTok{echo} \StringTok{"{-}{-}trim\_poly\_x"} \VariableTok{)} \DataTypeTok{\textbackslash{}}
        \AttributeTok{{-}{-}out1} \StringTok{"}\VariableTok{$TEMP\_DIR}\StringTok{/}\VariableTok{$\{BASENAME\}}\StringTok{\_trimmed.fastq"} \DataTypeTok{\textbackslash{}}
        \AttributeTok{{-}{-}json} \StringTok{"}\VariableTok{$TEMP\_DIR}\StringTok{/}\VariableTok{$\{BASENAME\}}\StringTok{\_fastp.json"} \DataTypeTok{\textbackslash{}}
        \AttributeTok{{-}{-}html} \StringTok{"}\VariableTok{$TEMP\_DIR}\StringTok{/}\VariableTok{$\{BASENAME\}}\StringTok{\_fastp.html"}

    \BuiltInTok{echo} \StringTok{"hello"} 

    \CommentTok{\# Run Salmon quantification}
    \ExtensionTok{salmon}\NormalTok{ quant }\AttributeTok{{-}i} \StringTok{"}\VariableTok{$SALMON\_INDEX}\StringTok{"} \DataTypeTok{\textbackslash{}}
        \AttributeTok{{-}l}\NormalTok{ A }\DataTypeTok{\textbackslash{}}
        \AttributeTok{{-}r} \StringTok{"}\VariableTok{$TEMP\_DIR}\StringTok{/}\VariableTok{$\{BASENAME\}}\StringTok{\_trimmed.fastq"} \DataTypeTok{\textbackslash{}}
        \AttributeTok{{-}o} \StringTok{"}\VariableTok{$TEMP\_DIR}\StringTok{/}\VariableTok{$\{BASENAME\}}\StringTok{\_salmon"} \DataTypeTok{\textbackslash{}}
        \AttributeTok{{-}{-}gcBias} \AttributeTok{{-}{-}seqBias} \AttributeTok{{-}{-}validateMappings}

\ControlFlowTok{done}

\CommentTok{\# Define output file with selected parameters}
\VariableTok{FINAL\_COUNT\_MATRIX}\OperatorTok{=}\StringTok{"}\VariableTok{$OUTPUT\_DIR}\StringTok{/gene\_count\_matrix\_}\VariableTok{$\{PARAMS\_STR\}}\StringTok{.csv"}

\CommentTok{\# Run R script to generate gene count matrix}
\ExtensionTok{Rscript} \StringTok{"}\VariableTok{$R\_SCRIPT}\StringTok{"} \StringTok{"}\VariableTok{$TEMP\_DIR}\StringTok{"} \StringTok{"}\VariableTok{$FINAL\_COUNT\_MATRIX}\StringTok{"}

\CommentTok{\# Remove temporary files}
\FunctionTok{rm} \AttributeTok{{-}rf} \StringTok{"}\VariableTok{$TEMP\_DIR}\StringTok{"}

\BuiltInTok{echo} \StringTok{"Pipeline complete. Final count matrix stored in }\VariableTok{$FINAL\_COUNT\_MATRIX}\StringTok{"}
\end{Highlighting}
\end{Shaded}

\hypertarget{references}{%
\section*{References}\label{references}}
\addcontentsline{toc}{section}{References}

\hypertarget{refs}{}
\begin{CSLReferences}{1}{0}
\leavevmode\vadjust pre{\hypertarget{ref-arora2020variability}{}}%
Arora, S., Pattwell, S. S., Holland, E. C., and Bolouri, H. (2020),
{``Variability in estimated gene expression among commonly used RNA-seq
pipelines,''} \emph{Scientific reports}, Nature Publishing Group UK
London, 10, 2734.

\leavevmode\vadjust pre{\hypertarget{ref-tong2020impact}{}}%
Tong, L., Wu, P.-Y., Phan, J. H., Hassazadeh, H. R., Tong, W., and Wang,
M. D. (2020), {``Impact of RNA-seq data analysis algorithms on gene
expression estimation and downstream prediction,''} \emph{Scientific
reports}, Nature Publishing Group UK London, 10, 17925.

\leavevmode\vadjust pre{\hypertarget{ref-vieth2019systematic}{}}%
Vieth, B., Parekh, S., Ziegenhain, C., Enard, W., and Hellmann, I.
(2019), {``A systematic evaluation of single cell RNA-seq analysis
pipelines,''} \emph{Nature communications}, Nature Publishing Group UK
London, 10, 4667.

\end{CSLReferences}



\end{document}
